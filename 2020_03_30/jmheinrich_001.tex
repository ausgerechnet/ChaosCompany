\documentclass{beamer}



%\usetheme{Marburg}
\usetheme{Berlin}

\usepackage{geometry}

\usepackage[english, ngerman]{babel}
\usepackage[T1]{fontenc}
\usepackage[utf8]{inputenc}

\usepackage{amsmath}                     % macht
\usepackage{amsfonts}                     %          mathe
\usepackage{amssymb}                     %                   mächtiger

\usepackage[]{units}
\usepackage{gensymb}
\usepackage{pifont}

\usepackage{appendixnumberbeamer}

\usepackage{multimedia}
\usepackage{color}
\usepackage{ifthen}
\usepackage{animate}

\usepackage{tikz}
\usetikzlibrary{positioning}
\usetikzlibrary{shadings}
\usetikzlibrary{fadings}
\usetikzlibrary{arrows}
\usetikzlibrary{decorations.markings}

\usepackage[framemethod=tikz]{mdframed}

\newcounter{angle} % ???
\setcounter{angle}{0} % ???

\setbeamercovered{transparent}
\setbeamercovered{dynamic}

\def\insertframetitle{}

%\definecolor{lkb}{RGB}{100,100,100}
\definecolor{lkb}{RGB}{0,162,212}
\definecolor{my_green}{RGB}{105, 184, 73}
\definecolor{my_red}{RGB}{244, 131, 107}


\setbeamercolor{secsubsec}{fg=lkb,bg=lkb}
\setbeamercolor{shadow}{fg=lkb,bg=lkb}

\setbeamercolor{block title}{use=structure,fg=white,bg=lkb}
\setbeamercolor{block body}{use=structure,fg=black,bg=lkb!20!white}

\setbeamercolor{block title example}{use=structure,fg=white,bg=my_green}
\setbeamercolor{block body example}{use=structure,fg=black,bg=my_green!20!white}

\setbeamercolor{block title alerted}{use=structure,fg=white,bg=my_red}
\setbeamercolor{block body alerted}{use=structure,fg=black,bg=my_red!20!white}



\setbeamertemplate{items}[circle]

\usepackage{epic}
\usepackage{graphicx}
\usepackage[percent]{overpic}
\usepackage{pict2e}

\beamertemplatenavigationsymbolsempty

% %--------------------- TO SHOW TOC AT BEGINNING OF EACH SECTION -----------------------
% \AtBeginSection[]
% {
%    \begin{frame}
%        \frametitle{Outline}
%        \tableofcontents[currentsection]
%    \end{frame}
% }
% %-----------------------------------------------------------------------------------------------------------

%--------------------- TO CUSTOMIZE THE APPEARANCE OF TOC -------------------------------
\setbeamertemplate{section in toc}{%
  {\color{black}\inserttocsectionnumber.}~{\color{lkb}\inserttocsection}}
\setbeamercolor{subsection in toc}{bg=white,fg=structure}
\setbeamertemplate{subsection in toc}{%
  \hspace{1.2em}\textcolor{lkb}{$\blacktriangleright$}~{\color{lkb}\inserttocsubsection\par}}
%-----------------------------------------------------------------------------------------------------------

%--------------------- TO DEFINE HEADER AND FOOTER -----------------------------------------
\setbeamertemplate{headline}
{
\leavevmode
  \begin{beamercolorbox}[wd=\paperwidth,ht=0.5cm,dp=0.2cm]{secsubsec}
    \raggedright
    \hspace*{0.2cm}
    {\sffamily\large\color{white}\insertsection\hfill\insertframetitle}
    \hspace*{0.2cm}
  \end{beamercolorbox}
  \tikz\draw[draw=none,top color=lkb,bottom color=lkb!20] (0,0) rectangle (\paperwidth,0.1);
}


\setbeamertemplate{frametitle}{}

\setbeamertemplate{footline}
{
  \leavevmode
     \begin{beamercolorbox}[sep=0.3cm,ht=1.8em,wd=\paperwidth]{}

%         \raisebox{-1.45mm}{\includegraphics[width=2cm]{logo_LKB.png}}
        \begin{tikzpicture}[>=stealth]
	            \draw [lkb,line width=1pt,path fading=east] (0,0) --(11,0);
 	 \end{tikzpicture}\hfill\raisebox{-0.05cm}{{\color{lkb}\insertframenumber/\inserttotalframenumber}}	
    \end{beamercolorbox}
}
%-----------------------------------------------------------------------------------------------------------

\makeatletter
\newlength\beamerleftmargin
\setlength\beamerleftmargin{\Gm@lmargin}
\makeatother

\graphicspath{{figures/}}


\usepackage{biblatex}

%-----------------------------------------------------------------------------------------------------------
%---- START OF DOCUMENT -------------------------------------------------------------------------
%-----------------------------------------------------------------------------------------------------------

\begin{document}

{
\begin{frame}[t,plain,noframenumbering]
	\vspace{0.7cm}
	\begin{mdframed}[tikzsetting={draw=lkb,fill=white,fill opacity=0.7, line width=4}, backgroundcolor=none, 							   leftmargin=0, rightmargin=0, innertopmargin=10]
		\vspace{0.5cm}
		{\Large{
  		\begin{center}Chaos Company\\[0.25cm]
                      \rule{5cm}{0.1cm}\\[0.25cm]
                      Status
        \end{center}
		}}
  		\vspace{0.75cm}
  		\centering
  		{\small{
  		\hspace{1cm} Johannes \hspace{1cm}\phantom{}\\
  		\phantom{}\hspace{1cm} 30/03/2020 \hspace{1cm}\phantom{}\\
  		}}
	\end{mdframed}

\end{frame}
}

\begin{frame}
\frametitle{Inhalt}
\tableofcontents[]
\end{frame}

%%%%%%%%%%%%%%%%%%%%%%%%%%%%%%%%%%%%%%%
\section{Spasspatent}
%%%%%%%%%%%%%%%%%%%%%%%%%%%%%%%%%%%%%%%

%-----------------------------------------------------------------------------------------------------------
\begin{frame}{W\"urfel auf Touchscreen}



\begin{block}{Motivation:}
Test des Patentprozesses: Versuche ein (nicht besonders gutes) Patent anzumelden um Erfahrung zu sammeln
\end{block}


\begin{exampleblock}{Idee 1:}
Preparierter W\"urfel welcher durch Touchscreen erkannt wird
\end{exampleblock}

\begin{alertblock}{Potential:}
Kann in jedem Brettspiel/W\"urfelspiel genutzt werden um die Haptik des W\"urfelns beizubehalten, und trotzdem mit Leuten zu spielen welche nicht im selben Raum sitzen.\\
Geringer technischer Aufwand, idealerweise auf bereits vorhanden Ger\"aten nutzbar, aber W\"urfel werden \"uber uns bezogen.
\end{alertblock}

\end{frame}
%-----------------------------------------------------------------------------------------------------------


%%%%%%%%%%%%%%%%%%%%%%%%%%%%%%%%%%%%%%%
\section{Pandemieprognose}
%%%%%%%%%%%%%%%%%%%%%%%%%%%%%%%%%%%%%%%

%-----------------------------------------------------------------------------------------------------------
\begin{frame}{Welche Gr\"ossen?}

\begin{exampleblock}{Idee 1:}
Geburtenrate steigt in 9Monaten, Erstgeborene, keine supercleveren Eltern
\end{exampleblock}

\begin{exampleblock}{Idee 2:}
Soziale Medien nach Neuentwicklungen durchk\"ammen\\
$\rightarrow$ Philipp. Erlaubnis zu nutzen?
\end{exampleblock}

\begin{exampleblock}{Idee 3:}
Verschiedene L\"ander, verschiedene G\"uter werden nachgefragt
\end{exampleblock}

\begin{alertblock}{Zweite Welle?}
Kommt eine zweite Welle? Selbst wenn nicht: Arbeits- und Sozialverhalten wird sich dauerhaft \"andern? Je l\"anger der momentane Zustand andauert, umso wahrscheinlicher.
\end{alertblock}

\end{frame}
%-----------------------------------------------------------------------------------------------------------


%%%%%%%%%%%%%%%%%%%%%%%%%%%%%%%%%%%%%%%
\section{Der Kniffelbot}
%%%%%%%%%%%%%%%%%%%%%%%%%%%%%%%%%%%%%%%

%-----------------------------------------------------------------------------------------------------------
\begin{frame}{Status}

\begin{center}
    \includegraphics[width=0.45\textwidth]{fig_kniffelbot_1}
    \includegraphics[width=0.45\textwidth]{fig_kniffelbot_2}
\end{center}

\end{frame}
%-----------------------------------------------------------------------------------------------------------

%-----------------------------------------------------------------------------------------------------------
\begin{frame}{Status}

\begin{center}
    \includegraphics[width=0.45\textwidth]{fig_kniffelbot_3}
    \includegraphics[width=0.45\textwidth]{fig_kniffelbot_4}
\end{center}

\end{frame}
%-----------------------------------------------------------------------------------------------------------


%-----------------------------------------------------------------------------------------------------------
\begin{frame}{Status}

\begin{center}
    \includegraphics[width=0.45\textwidth]{fig_kniffelbot_5}
    \includegraphics[width=0.45\textwidth]{fig_kniffelbot_6}
    
    \includegraphics[width=0.45\textwidth]{fig_kniffelbot_7}
    \includegraphics[width=0.45\textwidth]{fig_kniffelbot_8}
\end{center}

\end{frame}
%-----------------------------------------------------------------------------------------------------------

%-----------------------------------------------------------------------------------------------------------
\begin{frame}{Kosten bis jetzt}

\begin{center}
    \includegraphics[width=\textwidth]{fig_kniffelbot_kosten}
\end{center}

\end{frame}
%-----------------------------------------------------------------------------------------------------------


%-----------------------------------------------------------------------------------------------------------
\begin{frame}{Wie gehts weiter?..}

\begin{exampleblock}{Supermodulares CAD:}
Ein modulares Korsett in welches alle Bauteile gesteckt werden: raspberry pi, camera, touchscreen. Stichwort: Lego f\"ur Erwachsene.\\
$\rightarrow$ Martin und Rob?
\end{exampleblock}

\begin{exampleblock}{3D Druck:}
Was ist technisch m\"oglich? Anfertigung!\\
$\rightarrow$ Simon im Austausch mit Martin und Rob?
\end{exampleblock}

\begin{exampleblock}{Hardware Version 2:}
Kann das billiger werden? Konkrete Vorschl\"age.\\
$\rightarrow$ Johannes im Austausch mit ... ?
\end{exampleblock}

\begin{exampleblock}{Software Version 2:}
Datenbank, Homepage, Methoden verbessern.\\
$\rightarrow$ Johannes im Austausch mit Daniel
\end{exampleblock}

\end{frame}
%-----------------------------------------------------------------------------------------------------------



%%%%%%%%%%%%%%%%%%%%%%%%%%%%%%%%%%%%%%%
\section{Quarant\"anegadgets}
%%%%%%%%%%%%%%%%%%%%%%%%%%%%%%%%%%%%%%%

%-----------------------------------------------------------------------------------------------------------
\begin{frame}{Projekte}

\begin{exampleblock}{projekt\_a:}
Der Kniffelbot. Kunden: Sylvie will kniffeln \"uber Skype... Macht doch keinen Spass.
\end{exampleblock}

\begin{exampleblock}{projekt\_b:}
Das Minigew\"achshaus: Frische Kr\"auter aus der eigenen K\"uche.\\
\begin{itemize}
 \item Was ist auf dem Markt?
 \item Wer sind die Kunden und was darf es kosten?
 \item Welche Technologie? Raspberry Pi vs Arduino?
 \item Prototyp?!
\end{itemize}
\end{exampleblock}

\begin{exampleblock}{projekt\_c:}
Hab da was, gibt ne eigene Session...
\end{exampleblock}

\end{frame}
%-----------------------------------------------------------------------------------------------------------


%%%%%%%%%%%%%%%%%%%%%%%%%%%%%%%%%%%%%%%
\section{Firmenstruktur}
%%%%%%%%%%%%%%%%%%%%%%%%%%%%%%%%%%%%%%%

\begin{frame}{Divide and Conquer}

\begin{alertblock}{Idee:}
Subgruppen unserer Gruppe k\"onnen verschiedene Sachen besser und schlechter. Teile in kleinere Teams f\"ur jeweilige Aufgaben, in welchen die Teilnehmer konkurrieren, aber dennoch immer wieder zusammenarbeiten und sich gegenseitig helfen.
\end{alertblock}

\end{frame}
%-----------------------------------------------------------------------------------------------------------

\begin{frame}{Aufgabenverteilung}

\begin{exampleblock}{Aufgaben:}
Ein Quarant\"anegadget besteht aus: Idee, Finanzplan, Hardware, Mechanik, Software, Elektronik, Marketing, Gesamtorganisation, Patente(?). 
\end{exampleblock}

\begin{alertblock}{Verteilung:}
\begin{itemize}
 \item Ideenfindung/Marketing: Alle/Andi und alle?
 \item Finanzkontrolle: Vitali und Philipp?
 \item Hardware: Johannes und ?
 \item Mechanik/Patente: Martin und Rob und Simon?
 \item Software: Johannes und Daniel?
 \item Elektronik: Johannes und ?
\end{itemize}

\end{alertblock}


\end{frame}
%-----------------------------------------------------------------------------------------------------------


\begin{frame}{Wie gehts weiter?..}

\begin{alertblock}{Firma auf rechtliche Grundlage}
Umso fr\"uher umso besser! Stichworte: Finanzierung, Gewinnaufteilung, geistiges Eigentum, etc.
\end{alertblock}

\begin{alertblock}{Hausaufgabe}
\begin{itemize}
 \item Was kann jeder einbringen an Zeit, Geld, Wissen, Energie, Motivation, Kontakte, etc
 \item Was erwartet jeder kurz-, mittel- und langfristig?\\
 Beispiel Johannes: Ich will zuhause arbeiten, an Sachen die mir Spass machen, mit Leuten die ich mag, soviel wie ich grade lustig bin - und trotzdem damit Geld verdienen.
\end{itemize}
\end{alertblock}

\end{frame}
%-----------------------------------------------------------------------------------------------------------

\end{document}
%-----------------------------------------------------------------------------------------------------------
%---- END OF DOCUMENT ----------------------------------------------------------------------------
%-----------------------------------------------------------------------------------------------------------






